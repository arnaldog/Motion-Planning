\section{Estado del Arte}

%%% resumir porfavor %%%
El problema cl\'asico en el Path Planning se describe como:
\begin{quote}
dado un cuerpo r\'igido tridimensional y un conocido conjunto de obstaculos, la tarea de encontrar una ruta libre de colisiones desde una configuraci\'on incial a una objetivo. Adicionalmente, esta tarea debe ser completada en una cantidad razonable de tiempo.
\end{quote}
Este problema es conocido como el problema del movedor de pianos \cite{Swartz83}.

Existen paradigmas cl\'asicos para resolver Motion Planning como lo son los m\'etodos de campo potencial (potential field methods), m\'etodos de mapas de ruta como los m\'etodos de grafos de visibilidad y probabilisticos y los metodos de grillas \cite{Wang03} y \cite{Zhuang04} .

Los metodos de campo potencial primeramente descritos en \cite{Kathib85}, son metodos usados para evitar colisiones en base a sensores de proximidad. La idea es caracterizar al robot y los obstaculos como cargas positivas y al objetivo como carga negativa. La navegaci\'on ocurre a medida de las interacciones de repulsi\'on entre el agente y los obstaculos y la atracci\'on del agente con el objetivo. La interacci\'on total es $U(q) = U_{attr}(q) + U_{repul}(q)$ y cada paso es tomado sobre el gradiente negativo del potencial. El robot termina el movimiento cuando descubre que el gradiente es cero.

Los m\'etodos basados en mapas de rutas son aquellos que consideran como mapa a una estructura de datos usada para planear rutas subsecuentes mas r\'apidamente. La estructura de datos intenta capturar la conectividad y caracteristicas de la configuraci\'on del espacio del agente. Usando el mapa, un planeador (planner) puede encontrar rutas entre dos configuraciones donde lo primero es encontrar una ruta libre de colisiones desde una de las configuraciones al mapa de rutas y as\'i desde el mapa de rutas a la configuraci\'on de destino.

Dentro de los m\'etodos basados en mapas de rutas est\'an los mapas de visibilidad, primeramente estudiados por NJ Nilsson en 1969, los cuales consideran los nodos de los mapas como v\'ertices de un pol\'igono y dos nodos del grafo de visibilidad comparten un vertice si sus v\'ertices correspondientes son vistos por el otro. La linea de visi\'on puede estar dada por un sensor como la lectura de un sonar.

Tambi\'en, dentro de los m\'etodos basados en mapas, est\'an los mapas de rutas probabil\'isticos,  que se suelen ocupar cuando el n\'umero de grados de libertar del agente son moderadamente grandes y las aproximaciones para la planificaci\'on exacta se vuelven muy dif\'iciles de calcular o imposible.  Para esta clase de problemas se hace necesario recurrir a la utilizaci\'on de heur\'isticas para encontrar soluciones las que b\'asicamente consisten en chequear si la configuraci\'on del robot $q$ est\'a dentro del area libre de obst\'aculos.


El modelo com\'un en la investigaci\'on de Motion Planning ha sido desacoplar los problemas generales de rob\'otica relacionados con sensores, etc. resolviendo primero Global Path Planning y luego encontrar una trayectoria que pueda satisfacer las restricciones din\'amicas y seguir ese camino \cite{LaValle99}.


Tambi\'en se sugiere que si se est\'a preocupado por la inercia del robot, causando con ello un camino cinem\'atico planificado, se puede tambi\'en ejecutar el camino muy lentamente, para minimizar los efectos de la din\'amica. Aunque estas suposiciones son razonables en ciertas ocasiones, no pueden siempre ser justificadas. Incluso puede darse el caso de que el camino conseguido sea completamente no factible debido al entorno del objeto y sus limites de fuerzas y torques \cite{LaValle99}

Path Planning Problem puede ser dividido en dos grandes categor\'ias. La primera es crear un camino planeado a seguir dentro de un entorno est\'atico. La segunda es crear un plan para hacer un recorrido en un entorno din\'amico \cite{Goldman94}.

En un primer caso, se tiene, un punto de comienzo, un punto de meta (final o de llegada) y un set de regiones a evitar, todos conocidos incluso antes de que comienze a crearse un plan para crear el recorrido, mas a\'un, se asume que estas condiciones no cambiaran, con ello, se refiere a este caso como Global Path Planning \cite{Goldman94}.

En el segundo caso, parte de las regiones a evitar, o en su totalidad, son desconocidas, por ejemplo, situaciones de peligro solo se conocen cuando se est\'a en ese sector y dentro del rango de los sensores, esto modela de manera mas fiel los casos de la realidad, con ello, se har\'a referencia a este caso como Dynamic Path Planning

Ambas aproximaciones pueden parecer similares, pero de hecho, son muy diferentes. Por ejemplo en Global Path Planning, no se tiene mayor preocupacion sobre el tiempo que demora en formular una solucion, pues se puede determinar antes de que se comience a navegar, mientras que en el caso din\'amico la consideraci\'on del tiempo es mas cr\'itica, pues en cada momento se est\'a actualizando el plan a seguir, seg\'un cambien las condiciones. De aqu\'i se hace una distinci\'on entre una formulaci\'on en tiempo real y otra no en tiempo real \cite{Goldman94}

Otra gran diferencia entre ambos esquemas es la capacidad de decisi\'on, mientras que en el caso global es posible analizar si un camino es seguro, en el caso din\'amico puede darse el caso en el que no se puede concluir si un camino formulado es seguro o no. \cite{Goldman94}

Global Path Planning ha sido resuelto para dominios de robot y aeronaves a traves de una t\'ecnica llamada: ``Descomposici\'on de Collin'' (\textit{Collin's Descomposition}) \cite{Latombe91}, el cual consiste en dividir el entorno en muchas regiones mas peque\~nas y formulando un camino seguro e individual para cada uno de ellos, luego cuando las regiones esten completadas, se agrupan nuevamente estos sub-caminos para formal una solucion global, y en aquellos casos en que se requiere pasar por diferentes regiones, esta t\'ecnica puede ser realizadas varias veces.\cite{Goldman94}

Ciertamente cuando se requiere evitar zonas peligrosas, se pueden usar zonas potenciales la que consiste en asignar valores num\'ericos que permitan medir el nivel de seguridad, a aquellas regiones peligrosas y establecer circulos concentricos a ellos con valores intermedios, de esta forma, los robots pueden evitarlas y tomar decisiones como esquivarlas o regresar por donde vinieron. En especial este \'ultimo recurso no es una opcion para muchos problemas de Dynamic Path Planning como por ejemplo aeronaves que no tienen otra opci\'on, por limitaciones fisicas, de regresar rapidamente o ``dar la vuelta''.

Para solucionar ello, Goldman propone una soluci\'on llamada \textit{Subgoal Avoidance}, el que consiste en establecer sub-metas previas antes de dirigirse a la meta global, y con ellas evitar zonas riesgosas. Se representan las zonas de peligro como un espacio de dos dimensiones y se traza una linea perpendicular entre el camino original y el punto de riesgo y se mide la distancia $d$ entre la recta del camino original y el punto de riesgo, luego se crea una circunferencia de radio $2d$, la interseccion entre la circunferencia y el bisector perpendicular es un nuevo punto que ser\'a una submeta por la cual se debe pasar para evitar el peligro.\cite{Goldman94}


Desde la perspectiva de las heur\'isticas de resoluci\'on tambi\'en existen aproximaciones utilizando simulated annealing, algoritmos gen\'eticos, algoritmos basados en colonias de hormigas, en ejambre de particulas y en sistemas inmunes y combinaciones entre ellos.

Se proponen tambi\'en estrategias de planificaci\'on de rutas como el \textit{simulated annealing dissipative ant system}  \cite{Wang08} cuyo objetivo, en contraste con los sistemas de hormigas tradicionales, es utilizar simulated annealing para aumentar la capacidad global de b\'usqueda (exploraci\'on) del algoritmo donde solo las mejores rutas en cada iteraci\'on son actualizadas para incrementar la rapidez de aprendizaje del m\'etodo. A esto se le introduce la entrop\'ia en el sistema para salir de los m\'inimos locales.

Dentro de los algoritmos gen\'eticos hay investigaciones recientes utilizando algoritmos gen\'eticos elitistas aplicados al Global Path Plannning \cite{Tsai} los cuales consisten b\'asicamente en dos algoritmos gen\'eticos elitistas paralelos  para mantener una mayor diversidad y as\'i inhibir una convergencia prematura en comparaci\'on con los algoritmos gen\'eticos convencionales. La forma de generar soluciones iniciales factibles que sean suaves son a trav\'es de tecnicas de interpolaci\'on como la B-Spline c\'ubica \cite{Tsai} las cuales son continuas.

Dentro de lo algritmos basados en enjambre de part\'iculas  se proponen algoritmos especializados que garantizan convergencia para la optimizaci\'on \cite{Zenyu10}. La forma de representar el problema es configurar un mapa el cual conecta el nodo inicial con el nodo final donde cada nodo en la trayectoria es codificado como una par\'icula.  Entonces, una regi\'on particular ``activa'' para particulas es mapeada de acuerdo a la ubicaci\'on de los obstaculos. La poblaci\'on inicial de part\'iculas se genera dentro de la regi\'on activa para buscar la ruta \'optima. En el proceso de b\'usqueda, los coeficientes de aceleraci\'on y la inercia del algoritmo de optimizaci\'on basado en part\'iculas  es auto adaptativo ajustado y las part\'iculas invalidas son remplazadas por un \'optimo local o global en areas adyacentes.

El metodo de malla para resolver Global Path Planning muy usado en robots m\'oviles. Para este metodo se han implementado algoritmos como los basados en las direcciones principales para mejorar el desempe\~no de esta aproximaci\'on reduciendo el numero de nodos en la b\'usqueda \cite{Zhenyu101}.


Los algoritmos basados en colonias de hormigas son muy utilizados para resolver Global Path Planning, donde hay soluciones muy diversas como algoritmos de cobertura completa los cuales integran coberturas de sub areas locales con Global Path Planning \cite{Zang08} que utilizan descomposici\'on celular  donde los los agentes cubren una sub area a trav\'es de un movimiento de vaiv\'en donde la distancia entre cada sub area es refinada, la cual incluye conectividad, distancia restante y numero de obstaculos entre sub areas. A partir de aquello se calcula una matriz de distancias entre sub areas del entorno la cual es utilizada para Global Path Planning, de esta manera se utilizan los algoritmos basados en colonias de hormigas con dicha matriz para obtener la secuencia de optimizaci\'on de sub areas. Tambien se proponen mejoras para superar los defectos de la precocidad y el tiempo requerido para la construcci\'on  de la poblaci\'on inicial en el algoritmo tradicional al aumentar la colonia de hormigas para Global Path Planning. En \cite{Meijuan08} se propone un algoritmo mejorado para aumentar la colonia de hormigas donde las operaciones de cruce y mutaci\'on del algoritmo gen\'etico (GA) se utilizan en aumentar la optimizaci\'on de colonia de hormigas, y la funci\'on de probabilidad heur\'istico se agrega al proceso de la construcci\'on inicial de la poblaci\'on.


Al respecto tambi\'en existen soluciones basadas en algoritmos inmunes donde el algoritmo tiene capacidades de adaptaci\'on en base a la inmunidad del sistema inmunol\'ogico y para alcanzar el objeto de destino de manera segura y con cumplir con \'exito con su tarea a trav\'es de camino \'optimo. Los algortimos inmunes tienen resultados similares a los algoritmos gen\'eticos sin embargo tienen mejor desempe\~no cuando el entorno es m\'as complejo \cite{Xuanzi07} .

Considerando estas dos aproximaciones surgen modelos que consideran algoritmos basados en colonias de hormigas y redes inmunes \cite{Mingxin08}  donde el mecanismo de estimulaci\'on y supresi\'on entre ant\'igeno y anticuerpo es usado para encontrar la ruta. la cual resuelve el modelamiento complejo de los sistemas basados en hormigas, lo cual mejora la eficiencia en encontrar la ruta \'optima. El algoritmo basado en hormigas es usado para buscar en la red de anticuerpos, lo cual aumenta el efecto en la planificaci\'on de ruta. Los autores concluyen que estos algoritmos est\'an caracterizados por tener una r\'apida convergencia y la planeaci\'on de rutas cortas, el cual resuelve planificaci\'on en ambientes complejos \cite{Mingxin08}.
