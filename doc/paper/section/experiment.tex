\section{Experimentos}
Los experimentos son realizados sobre un conjunto de instancias que caracterizan situaciones de diferentes dificultades para el agente. Uno de lo objetivos de realizar la experimentaci�n sobre distintas instancias es el de poder encontrar los par�metros adecuados para cada tipo de instancia (nivel de dificultad) y establecer un razonamiento adecuado para la futura utilizaci�n del algoritmo (heur�sticas). Otro objetivo muy importante es medir la calidad de las soluciones encontradas y poder hacer un breve an�lisis de �stas conjunto con los tiempos de ejecuci�n del algoritmo y evaluar el tiempo de convergencia respecto a los par�metros. 

\subsection{Organizaci�n de los Experimentos}
Se realiza por cada instancia una serie de experimentos probando diferentes par�metros, midiendo la calidad de las soluciones y observando el tiempo de ejecuci�n del algoritmo. Otro dato importante de observar es la convergencia del valor de la funci�n objetivo durante las iteraciones del algoritmo lo que permite ver a posteriori el comportamiento de la velocidad del enjambre de part�culas.

%%% Incluir Instancias a resolver %%%%
Tablas con experimentos 

\begin{table}[ht]
	\begin{center}
	\begin{tabular}{lllllllll}
	\toprule
	Instancia	& $|S|$	& $\alpha$	&$\omega$	& $\phi_g$	& $\phi_p$	& $q$	& M�todo	& Muestras	\\
	\hline
	Bugtrap		& 10	& 2			& 1			& 10		& 20		& 100	& Hermit�   & 100\\ 	
	Bugtrap		& 10	& 2			& 0			& 10		& 20		& 100	& Bezier    & 100\\ 
	T			& 10	& 0			& 1			& 10		& 20		& 100	& Hermit�   & 100\\ 
	T			& 10	& 0			& 0			& 10		& 20		& 100	& Bezier    & 100\\ 
	Rooms		& 10	& 0			& 1			& 10		& 20		& 100	& Hermit�   & 100\\ 
	Rooms		& 10	& 0			& 0			& 10		& 20		& 100	& Bezier    & 100\\ 
	Complex		& 500	& 2			& 1			& 50		& 20		& 100	& Bezier    & 100\\
	Complex		& 500	& 2			& 0			& 50		& 20		& 100	& Bezier    & 100\\
	\bottomrule
	\end{tabular}
	\caption{Parametros PSO utilizados seg�n instancia}
	\end{center}
\end{table}




	
\begin{figure}
  \centering
  \subfloat[Bugtrap]{\label{fig:bugtrap}\includegraphics[width=0.20\textwidth]{\imagepath{bugtrap.png}}}                
  \subfloat[Complex]{\label{fig:complex}\includegraphics[width=0.20\textwidth]{\imagepath{complex.png}}} \\
  \subfloat[T]{\label{fig:t}\includegraphics[width=0.20\textwidth]{\imagepath{t.png}}}
  \subfloat[Rooms]{\label{fig:rooms}\includegraphics[width=0.20\textwidth]{\imagepath{rooms.png}}}
 
  \caption{Instancias Global Path Planning}
  \label{fig:animals}
\end{figure}







%\begin{verbatim}
%./psomp.bin -iteraciones 50  -particulas 500  -alpha 2 -omega 2  -phip 20 -phig50 -map %../psomp/maps/rooms_easychico.dat  -pivotes 8 -mode bezier
%\end{verbatim}


\begin{table}[ht]
	\begin{center}
	\begin{tabular}{lllll}
	\toprule
	Instancia & Calidad & Calidad Promedio & Tiempo & Entregable \\
	T Chico	  &  32		 & 12		& 3124	   & 230 \\
	\bottomrule
	\end{tabular}
	\caption{Resultados PSO utilizados seg�n instancia}
	\end{center}
\end{table}


\subsection{Resultados}

