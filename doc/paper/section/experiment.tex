\section{Experimentos}
Los experimentos son realizados sobre un conjunto de instancias que caracterizan situaciones de diferentes dificultades para el agente. Uno de lo objetivos de realizar la experimentaci�n sobre distintas instancias es el de poder encontrar par�metros adecuados o criterios ad-hoc para cada tipo de instancia (nivel de dificultad) y as� establecer un razonamiento adecuado para la futura utilizaci�n del algoritmo (heur�sticas). Otro objetivo muy importante es medir la calidad de las soluciones encontradas y poder hacer un breve an�lisis de �stas conjunto con los tiempos de ejecuci�n del algoritmo y evaluar el tiempo de convergencia respecto a los par�metros. 

La b�squeda de par�metros adecuados significa realizar an�lisis exploratorio de datos. Para analizar los datos, pueden ser utilizadas diferentes t�cnicas, una t�cnica a considerar es estudiar los datos desde la perspectiva de un modelo, donde los par�metros del problema son las variables explicativas (o caracter�sticas) del modelo y medidas como el fitness o el tiempo de ejecuci�n son la salida del modelo.

La idea de estudiar el fitness explicado a trav�s de los par�metros puede ser considerada porque la relaci�n entre ambos no es fuertemente causal, dado que el algoritmo inicializa pseudo aleatoreamente a las part�culas y existen incertidumbres asociadas al valor final de la funci�n objetivo lo que permite realizar an�lisis estad�stico para estudiar tal situaci�n.

Con un modelo estad�stico asociado al problema es posible realizar simulaciones y as� ampliar la perspectiva desde el punto de vista de las ejecuciones dado que la obtenci�n de un conjunto variado de soluciones constituye consumir largos per�odos de tiempo.

Es importante se�alar que el objetivo del an�lisis no es el de determinar o proponer un modelo de regresi�n u otra clase de ajuste, sino el de estudiar el comportamiento generando modelos emp�ricos como las redes neuronales artificiales.

\begin{figure}
\begin{center}
\includegraphics[width=\columntextwidth]{\imagepath{ann.png}} 
\caption{Modelo de red neuronal para parametros vs salida del PSO}
\end{center}
\end{figure}
El uso de modelos emp�ricos es m�s adecuado dado que estos pueden aproximar funciones no lineales y provienen directamente de la data, sin embargo no son expl�citos, lo que no permite una visualizaci�n instantanea entre la salida y la variable explicativa a utilizar.


\subsection{Organizaci�n de los Experimentos}
Se realiza por cada instancia una serie de experimentos probando diferentes par�metros, midiendo la calidad de las soluciones y observando el tiempo de ejecuci�n del algoritmo. Otro dato importante de observar es la convergencia del valor de la funci�n objetivo durante las iteraciones del algoritmo lo que permite ver a posteriori el comportamiento de la velocidad del enjambre de part�culas.

El procedimiento de la experimentaci�n de este algoritmo consiste en considerar el siguiente escenario.

\paragraph{Ejemplo de ejecuci\'on del algoritmo} El algoritmo especifica todos los par\'ametros antes mencionados
y adem\'as especificar el modo de interpolaci\'on y la instancia.

\begin{verbatim}
./psomp.bin -iteraciones 50  -particulas 500 
	-alpha 2 -omega 2  -phip 20 -phig50
	-map %../psomp/maps/rooms_easychico.dat
	-pivotes 8 -mode bezier
%\end{verbatim}


%%% Incluir Instancias a resolver %%%%
\begin{table}[ht]
	\begin{center}
	\begin{tabular}{lllll}
	\toprule
	Entradas		& Valores		&			&					&				\\
	\hline
	Instancia		& \scriptsize bugtrap		& \scriptsize T		& \scriptsize rooms\_easychico	& \scriptsize complexchico  \\
	\hline
	$|S|$			& 50			& 250		&					&				\\ \hline
	$\alpha$		& 2				& 2.5		& 3					&				\\ \hline
	$\omega$		& 0.7			& 1.5		&					&				\\ \hline
	$\phi_g$		& 0.6			& 1			& 1.4				&				\\	 \hline		
	$\phi_p$		& 0.6			& 1			& 1.4				&				\\ \hline
	$q$				& 1				& 2			& 4					& 6				\\ \hline
	M�todo			& \scriptsize Hermit�		& \scriptsize Bezier	&					&				\\ \hline
	Iteraciones 	& 5				& 10		& 30				&				\\ \hline
	Ejecuciones		& 3				&			&					&				\\ \hline
	Bases Hermite	& 1				& 10		& 20				&				\\ 
	\bottomrule
	\end{tabular}
	\caption{Parametros PSO utilizados seg�n instancia}
	\end{center}
\end{table}

En total son 3456 ejecuciones para el PSO utilizando Bezier y 7588 usando Hermite
Cada configuracion se realiza 3 veces, lo que significa un total de 220368 ejecuciones del algoritmo.








	
%\begin{figure}
%  \centering
%  \subfloat[Bugtrap]{\label{fig:bugtrap}\includegraphics[width=0.20\textwidth]{\imagepath{bugtrap.png}}}                
%  \subfloat[Complex]{\label{fig:complex}\includegraphics[width=0.20\textwidth]{\imagepath{complex.png}}} \\
%  \subfloat[T]{\label{fig:t}\includegraphics[width=0.20\textwidth]{\imagepath{t.png}}}
%  \subfloat[Rooms]{\label{fig:rooms}\includegraphics[width=0.20\textwidth]{\imagepath{rooms.png}}}
 
%  \caption{Instancias Global Path Planning}
%  \label{fig:animals}
%\end{figure}


\begin{table}[ht]
	\begin{center}
	\begin{tabular}{lllll}
	\toprule
	Instancia & Calidad & Calidad Promedio & Tiempo & Entregable \\
	T Chico	  &  32		 & 12		& 3124	   & 230 \\
	\bottomrule
	\end{tabular}
	\caption{Resultados PSO utilizados seg�n instancia}
	\end{center}
\end{table}





\subsection{Resultados}

