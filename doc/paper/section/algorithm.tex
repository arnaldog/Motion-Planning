\section{Descripci�n del Algoritmo}

%%% REPRESENTACION PARA EL ENJAMBRE
	
El enjambre de part�culas corresponde al conjunto de $S$ part�culas, donde part�cula es una eventual soluci�n y corresponde a una ruta a la que se le calcula una velocidad de movimiento respecto a su mejor posici�n hist�rica y respecto a la mejor posici�n global determinada la inicializaci�n y posteriormente determinado durante la iteraci�n.
El objetivo del Global Path Planning PSO es que cada part�cula tome movimientos relativos al mejor del grupo y a su propia informaci�n de mejor posici�n.


	\begin{enumerate}
	\item El mapa est� representado como una matriz  con coeficientes binarios
	\[Map = \left[
		\begin{array}{ccc}
		(1,1)  & \cdots & (1, m) \\
		\vdots & \ddots & \vdots \\
		(n, 1) & \cdots & (n, m)  \\
		\end{array}\right]
	\]

	\item Una part�cula corresponde a una soluci�n del problema, es decir una ruta $r$ de longitud $l$
	\[ r = \{ (p_{1x}, p_{1y}) , \cdots, ( p_{lx}, p_{ly}) \}, \quad |r| = l \]
	\item Todas las combinaciones de rutas posibles dentro del mapa representan el dominio de cada ruta
	\item La longitud de $k$ la velocidad es fija, y cada punto se calcula con $k$ incrementos de
		   $ \delta = \lfloor l/k \rfloor$ sobre la ruta $p$.
	\[ v = \{ (p_{\delta x}, p_{\delta y}) , \cdots, ( p_{k\delta x}, p_{k \delta y}) \}, \quad |v| = k  \]
	\end{enumerate}
	
	Representaci�n
	\begin{enumerate}
	\item Cada ruta $p$ ser� construida pseudo aleatoreamente 
	\item La velocidad de la part�cula en el inicio corresponder� a sus pivotes
	\item Para evaluar la velocidad entre las part�culas se actualizan los pivotes de las
		  velocidades en vez de la posici�n.	
    \item La actualizaci�n de la posici�n corresponde a la generaci�n de nuevas rutas
    	  que pasen por los pivotes antes calculados.
	\end{enumerate}
	
	Funci�n objetivo
	\begin{equation}
		\min F(r) = lenght(r) + C(1 + lenght(r)^\alpha)
	\end{equation}
	
	\begin{enumerate}
	\item $lenght(r)$ es el largo de la ruta
	\item $ C$ es el n�mero de colisiones
	\item $\alpha$ es el nivel de penalizaci�n por rutas cortas con colisiones
	\end{enumerate}
	
	{Posici�n y velocidad de la part�cula}
	\begin{eqnarray}
		r = \{ p_1, \ldots, p_l \} \\
		v = \{ v_1, \ldots, v_k \} 
	\end{eqnarray}
	
	\begin{enumerate}
	\item La ruta $r$ se v� generando pseudo aleatoreamente 
	\item Por ende, todas las rutas tienen largos distintos.
	\item Sin embargo, la longitud del vector velocidad es constante dado como par�metro
	\item En este caso difiere respecto a la velocidad respecto a la del PSO original.
	\end{enumerate}
	
	{Actualizac�on de velocidad y posici�n de la part�cula}
	\begin{eqnarray}
	 v_{i+1} = \omega v_i + \rho_g\phi_g (g - x_i) + \rho_p\phi_g (p_i- x_i) \\
	  r_{i+1} =  r_i +  v_i 
	\end{eqnarray}
	
	Los par�metros son:
	\begin{enumerate} 
	\item $\omega$ que pondera el efecto de la velocidad actual
	\item $\phi_g$ par�metro para favorecer la explotaci�n
	\item $\phi_p$ par�metro para favorecer la exploraci�n
	\item $\rho_g, \rho_p \sim U(0,1)$
	\item $g$ la mejor posici�n global
	\item $p_i$ la mejor posici�n de la part�cula $i$
	\item $r_i$ posici�n de la part�cula $i$ (la ruta $i$).
	\end{enumerate}
	
	{Actualizac�on de velocidad y posici�n de la part�cula}
	\begin{eqnarray}
	 v_{i+1} = \omega v_i + \rho_g\phi_g (v(g) - v(r_i)) + \rho_p\phi_g (v(p_i) - v(r_i)) \\
	  r_{i+1} = \text{Construir ruta a partir de }v^{t+1}
	\end{eqnarray}
	
	Se calculan las diferencias entre los pivotes (velocidades)
	\begin{enumerate} 
	\item $v(g) - v(r_i)$, la velocidad se calcula como
	\item $v(p_i) - v(r_i)$ 
	\end{enumerate}
	Donde la velocidad corresponde a un punto en el plano con componentes $x,y$
	\begin{eqnarray}
		v_x &=& \omega v_i_x + \rho_g\phi_g (v(g_x) - v(r_x)) + \rho_p\phi_g (v(p_x) - v(r_x)) \\
		v_y &=& \omega v_i_x + \rho_g\phi_g (v(g_y) - v(r_y)) + \rho_p\phi_g (v(p_y) - v(r_y))
	\end{eqnarray}

%%%% OPERADORES DE RUTA


%%% ALGORITMO PSO
Inicializaci�n del enjambre
\begin{algorithm}[H]% <- necesario
\SetLine \KwData{$S, \omega, \Phi_g, \Phi_p$}
\KwResult{Mejor particula $s_i$. }
\linesnumbered
\SetVline
\For{$i=1$ \KwTo $S$ }
{
	$r_i = $ \text{Inicializar posicion aleatoreamente}\;
	$p_i \leftarrow r_i$\;
	$v_i = $ \text{Inicializar velocidad aleatoreamente}\;
	\If{$f(p_i) < f(g)$}{
		$g \leftarrow p_i$
	}
	
}
\end{algorithm}

Iteraci�n del enjambre
\begin{algorithm}[H]% <- necesario
\SetLine \KwData{$S, \omega, \Phi_g, \Phi_p, I$}
\KwResult{Mejor particula $g$. }
\linesnumbered
$i = 0$
\While{$ i < I $}{
	\For{$i=1$ \KwTo $S$}
	{
		$ r_p, r_g \sim U(0,1)$\;
		$ v_i \leftarrow \omega v_i + \Phi_p r_p (p_i - r_i) + \Phi_g r_g(g - r_i)$\;
		$ x_i \leftarrow x_i + v_i $\;
		\If{$ f(r_i) < f(p_i)$}
		{
			$p_i \leftarrow r_i$\;
			\If{$f(p_i) < f(g)$}
			{
				$g \leftarrow p_i$\;
			}
		}
	}
	$i=i+1$\;
}
\Return $g$ \;
\end{algorithm}